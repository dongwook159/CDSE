\documentclass{article}
\usepackage{amsfonts,amsmath,amssymb}
\usepackage{bm}
\usepackage{geometry}
\usepackage{graphicx}
\usepackage{placeins}
\usepackage{tikz}
\usetikzlibrary{arrows}

\providecommand{\mat}[1]{ {\rm\bf #1} }
\providecommand{\pvd}[2]{\ensuremath{\begin{pmatrix} #1 \\ #2 \end{pmatrix}}}
\providecommand{\pdiff}[2]{\ensuremath{\frac{\partial #1}{\partial #2}}}

\begin{document}
{\centering Correspondence for GP and RBF reconstruction\\}
Let $\lbrace x_1,\ldots,x_N\rbrace$ be a set of equispaced points in $\mathbb{R}$ with separation $\Delta x$. Define the cell averages
\begin{equation}
  \overline{u}_i = \frac{1}{\Delta x}\int\limits_{x_{i-1/2}}^{x_{i+1/2}}u(x)dx.
\end{equation}
To distinguish between the two approaches, let $\widetilde{u}(x)$ denote the RBF approximant, and $\hat{u}$ denote the GP approximant. Finally, let $\phi(x,y)$ denote the basis function for RBF reconstruction, and $K(x,y)$ denote the GP kernel. Both are symmetric, such that each is only a function of $\left|x-y\right|$, though the arguments will be kept distinct in the following.

% --------------------------------------------------
\section*{The RBF approach}
From the RBF approach in Section (3.2) of Jared and Hyoseon's writeup, the RBF approximant is
\begin{equation}
  \widetilde{u}(x) = \sum\limits_{j=1}^n \alpha_j\phi(x,x_j),
  \label{eq:rbfApp}
\end{equation}
where the coefficients $\alpha_j$ are such that
\begin{equation}
  \frac{1}{\Delta x}\int\limits_{x_{i-1/2}}^{x_{i+1/2}} \widetilde{u}(x)dx = \overline{u}_i
  \label{eq:rbfCond}
\end{equation}
is satisfied for $i=1,\ldots,N$. Subtituting \eqref{eq:rbfApp} into \eqref{eq:rbfCond} yields a linear system for the coefficients, $\mat{A}\bm{\alpha} = \overline{\mat{u}}$, where the matrix has the entries
\begin{equation}
  A_{ij} = \frac{1}{\Delta x}\int\limits_{x_{i-1/2}}^{x_{i+1/2}} \phi(x,x_j)dx.
\end{equation}
Entry $A_{ij}$ is the basis function centered on $x_j$ averaged over the cell with center $x_i$.

% --------------------------------------------------
\section*{The GP approach}
In the GP framework the posterior mean function, in somewhat abused notation, is
\begin{equation}
  \hat{u}(x) = \mat{T}^T(x)\mat{C}^{-1}\overline{\mat{u}},
\end{equation}
where the covariance matrix has entries
\begin{equation}
  C_{ij} = \frac{1}{\Delta x^2}\int\limits_{x_{i-1/2}}^{x_{i+1/2}}\int\limits_{x_{j-1/2}}^{x_{j+1/2}} K(x,y) dydx,
\end{equation}
and the prediction vector is defined as
\begin{equation}
  T_j(x) = \frac{1}{\Delta x}\int\limits_{x_{j-1/2}}^{x_{j+1/2}} K(x,y)dy.
\end{equation}
Expanding this out a bit allows the GP approximant to be written as
\begin{equation}
  \hat{u}(x) = \sum\limits_{j=1}^n \beta_jT_j(x),
  \label{eq:gpApp}
\end{equation}
where $\bm{\beta} = \mat{C}^{-1}\overline{\mat{u}}$.

% --------------------------------------------------
\section*{Equivalence}
These approaches can be made equivalent to each other, that is, they can be made to yield the same approximant. Let the RBF basis function be defined in terms of the GP kernel as
\begin{equation}
  \phi(x,y) = \frac{1}{\Delta x}\int\limits_{y-\frac{\Delta x}{2}}^{y+\frac{\Delta x}{2}} K(x,s) ds,
  \label{eq:phipsi}
\end{equation}
where $s$ is simply a dummy variable of integration. This is very similar to the entries in \eqref{eq:rbfCond}, but now the arguments remain continuous. A quick change of variables gives that $\phi(x,y)=\phi(y,x)$ still holds. Now the RBF approximant becomes
\begin{eqnarray}
  \widetilde{u}(x) &=& \sum\limits_{j=1}^n \alpha_j\phi(x,x_j) \\
  &=& \frac{1}{\Delta x}\sum\limits_{j=1}^n \alpha_j\left(\int\limits_{x_{j-1/2}}^{x_{j+1/2}} K(x,s) ds\right) \\
  &=& \sum\limits_{j=1}^n \alpha_jT_j(x).
  \label{eq:rbfApp2}
\end{eqnarray}
Next, inserting this form of $\phi$ into \eqref{eq:rbfCond} yields
\begin{eqnarray}
  A_{ij} &=& \frac{1}{\Delta x}\int\limits_{x_{i-1/2}}^{x_{i+1/2}} \phi(x,x_j)dx \\
  &=& \frac{1}{\Delta x^2}\int\limits_{x_{i-1/2}}^{x_{i+1/2}} \left(\int\limits_{x_{j-1/2}}^{x_{j+1/2}} K(x,s) ds\right) dx \\
  &=& C_{ij}.
\end{eqnarray}
Therefore, if $\phi$ and $K$ satisfy \eqref{eq:phipsi}, then $\bm{\alpha}=\bm{\beta}$ will hold, and from equations \eqref{eq:gpApp} and \eqref{eq:rbfApp2} we have $\widetilde{u}(x) = \hat{u}(x)$.

\end{document}
